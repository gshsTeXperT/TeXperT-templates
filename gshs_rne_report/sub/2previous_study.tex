\section{선행 연구}
선행 연구 작성.

Overleaf에서는 왼쪽 상단의 Menu 버튼을 클릭하여 Compiler를 XeLaTeX으로 설정한다. 또, 함초롬 글꼴을 사용하기 위해, 한글과컴퓨터 다운로드에서 함초롬체를 다운로드하고 압축을 푼 뒤, fonts 폴더를 새로 만들어 그곳에 HANBatang.ttf, HANBatangB.ttf, HANDotum.ttf, HANDotumB.ttf를 이름을 바꾸지 말고 그대로 업로드하고, \texttt{main.tex}의 맨 윗줄을 \texttt{\textbackslash documentclass[overleaf]\{gshs-rne-report\}}와 같이 \texttt{overleaf} 옵션을 넘겨주면 된다. 로컬에서 작업하는 경우에도 컴퓨터에 폰트 설치가 안 되거나 폰트를 TeX에서 찾지 못하면, 앞과 같은 방법으로 fonts 폴더에 있는 폰트를 사용하게 할 수 있다.