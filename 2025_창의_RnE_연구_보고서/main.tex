\documentclass[10pt]{article}

\def\engtitle#1{\gdef\@engtitle{#1}}
\def\engauthor#1{\gdef\@engauthor{#1}}

\usepackage[utf8x]{inputenc}
\usepackage{fontspec}
\setromanfont{TIMES}[Path=./fonts/,Extension=.ttf,BoldFont=*BD,ItalicFont=*I,BoldItalicFont=*BI]
\usepackage{kotex}
\setmainfont{HANBatang.ttf}[Path=./fonts/,BoldFont=*B,WordSpace={1,1,0}]
\newhangulfontfamily{\hdotum}{Dotum}[Path=./fonts/,Extension=.ttf]
\newfontfamily{\dotum}{Dotum}[Path=./fonts/,Extension=.ttf]
\setsanshangulfont{Malgun}[Path=./fonts/,Extension=.ttf,BoldFont=*B]
\setsansfont{Malgun}[Path=./fonts/,Extension=.ttf,BoldFont=*B]
\usepackage[a4paper,left=25mm,right=25mm,top=25mm,bottom=20mm]{geometry}
%\usepackage{mathtools}
%\numberwithin{equation}{section}
\usepackage{amsmath}
\usepackage{amsthm}
\usepackage{amssymb}
%\usepackage{unicode-math}

\DeclareMathOperator*{\argmax}{arg\,max}
\DeclareMathOperator*{\argmin}{arg\,min}

\usepackage{amsfonts}
\usepackage{xcolor}
\usepackage{indentfirst}
\usepackage{tabularray}
\usepackage{siunitx}
\usepackage{enumitem}
\usepackage[linesnumbered, ruled]{algorithm2e}
\usepackage{amstext,bm}
\sisetup{group-separator = {,}}
\usepackage{titlesec}
\usepackage{hyperref}
\titleclass{\part}{straight}
\titleformat{\part}[hang]
{\normalfont\bfseries\fontsize{15}{15}\selectfont}{\thepart}{0em}{.~\,}
\titlespacing*{\part}{0pt}{1.5ex plus 1ex minus .2ex}{1.3 ex plus .2ex}
\renewcommand{\thepart}{\arabic{part}}

\renewcommand\thesection{\Roman{section}} 
\renewcommand\thesubsection{\arabic{subsection}} 
\titleformat{\section}[block]{\normalfont\fontsize{12}{12}\selectfont\bfseries}{\fontspec[Path=./fonts/]{HANBatangB.ttf}\selectfont\thesection.}{0.5em}{}
\titlespacing*{\section}{0pt}{2.5ex plus 1ex minus .2ex}{1.3ex plus .2ex}
\titleformat{\subsection}[block]{\fontsize{11}{11}\selectfont\bfseries}{\fontspec[Path=./fonts/]{HANBatangB.ttf}\selectfont\thesubsection.}{0.5em}{}
\titlespacing*{\subsection}{1ex}{1.5ex plus 0.6ex minus .3ex}{0.8ex plus .2ex}
\titleformat{\subsubsection}[block]{\hspace{1ex}\bfseries}{\fontspec[Path=./fonts/]{HANBatangB.ttf}\selectfont\thesubsubsection.}{0.5em}{}
\titlespacing*{\subsubsection}{0.5ex}{1.2ex plus 0.5ex minus .2ex}{0.5ex plus .1ex}

\usepackage{tikz}
\newcommand*{\slant}[2][76]{%
	\begingroup
	\hspace*{-1em}
	\sbox0{#2}%
	\pgfmathsetlengthmacro\wdslant{\the\wd0 + cos(#1)*\the\wd0}%
	\leavevmode
	\hbox to \wdslant{\hss
		\tikz[baseline=(X.base),inner sep=0pt,transform canvas={xslant=cos(#1)}]\node(X){\usebox0};%
		\hss
		\vrule width 0pt height\ht0 depth\dp0 %
	}%
	\endgroup  
}

\let\u\underline
\let\b\textbf
\let\i\textit
\newcommand{\til}{\,\textasciitilde\,}
\setlength{\parindent}{0pt}
\pagestyle{plain}
\usepackage{etoolbox}
\usepackage{setspace}
\setstretch{1.4}
\everydisplay\expandafter{%
	\the\everydisplay%
	\renewcommand{\baselinestretch}{1}\selectfont%
}

\makeatletter
\g@addto@macro\normalsize{%
    \setlength\abovedisplayskip{10.0pt plus 1.5pt minus 3.0pt}%
    \setlength\belowdisplayskip{10.0pt plus 1.5pt minus 3.0pt}%
    \setlength\abovedisplayshortskip{0.0pt plus 2.0pt}%
    \setlength\belowdisplayshortskip{6.0pt plus 3.0pt minus 3.0pt} 
}%


\providecommand*\setfloatlocations[2]{\@namedef{fps@#1}{#2}}
\setfloatlocations{figure}{htbp}
\setfloatlocations{table}{htbp}
\makeatother

\definecolor{brownbg}{HTML}{8f7d41}
\definecolor{brownfg}{HTML}{7a5721}
\definecolor{bluefg}{HTML}{437fc1}

\newcommand\teacherchecklist

\renewenvironment{abstract}
{\begin{center}\fontsize{12}{12}\selectfont\bfseries 가. 국문 초록\end{center}
	\hrule
	\setlength{\parindent}{7pt}\setlength{\parskip}{0.25\baselineskip plus 2pt}\par}
{\vspace{6pt}\hrule}

\newenvironment{paper}
{\setlength{\leftskip}{1ex}
	\setlength{\parindent}{7pt}\setlength{\parskip}{0.25\baselineskip plus 2pt}
	\setcounter{table}{0}}
{\clearpage}

\usepackage{comment}
\usepackage{subcaption}
\usepackage{tikz}
\usepackage{booktabs}
\usepackage{comment}
\usepackage[capitalize]{cleveref}
\crefname{section}{\S}{\S}
\Crefname{section}{\S}{\S}

\graphicspath{ {./figures} }
\begin{document}
\bgroup
%{\fontsize{18}{20}\selectfont\bfseries [작성서식]\fontsize{10}{10}\selectfont}\bigskip
	
	%{\dotum\hdotum\fontsize{9}{9}\selectfont\color{bluefg}~【서식 4.2.2.】 중간성과공유회 개최 시 R\&E지원센터, 연구보고서 평가 시 창의재단 PMS 전체 보고서 취합접수}
	
	\begin{center}\fontsize{16}{16}\selectfont\b{\u{2025년 과학영재 창의연구(R\&E) 보고서}}\end{center}\vspace*{1ex}
	\begin{tblr}{%
			width=\textwidth,
			columns={font=\bfseries},
			colspec={|[1.5pt]Q[gray9,fg=black,c,m,wd=80pt]|X[3,c]|X[5,c]|X[6,c]|X[4,c]|[1.5pt]},
            cell{1}{2}={c=2}{l},
            cell{1}{4}={}{gray9},
			cell{2-5}{2}={c=4}{l},
			cell{6-7}{2}={c=4}{c},
			cell{8}{1}={r=6}{c},
            cell{8}{2}={}{gray9},
            cell{8}{3}={c=3}{c},
			cell{9}{2}={r=5}{c},
			row{6-7}={ht=2.5em},
            row{9-13}={ht=1.5em},
			row{9}={gray9, fg=black},
            vline{3-5}={8-13}{dotted},
            hline{10-13}={1-5}{dotted},
			cell{14}{1}={c=5}{c,white, fg=black},
			colsep=0pt,row{4-5,8}={abovesep=0pt,belowsep=0pt},
			hspan=minimal,
			stretch=1.1}
		\hline[1.5pt]
        \SetCell[r=1]{c}학교명 & {\fontsize{10}{10}\selectfont 경기과학고등학교} & & 과제 관리번호 & \\ \hline
		\SetCell[r=2]{c}연구과제명 & {\fontsize{10}{10}\selectfont (국문) 국문 제목} & & & \\ \hline
		& {\fontsize{10}{10}\selectfont (영문) 영문 제목} & & & \\ \hline
		{연구유형\\\vskip-1ex(√)} & \begin{tblr}{vline{2-5}={1-2}{dotted},hline{2}={1-5}{dotted},colspec={X[c]X[c]X[c]X[c]X[c]},row{1}={gray9,fg=black}} 자율주제R\&E & 지정주제R\&E & 다년도R\&E & IP-R\&E & 글로벌R\&E \\
			√ & & & & \end{tblr} & & & & & \\\hline
		{연구분야\\\vskip-1ex(√)} & {\begin{tblr}{vline{2-8}={1-2}{dotted},hline{2}={1-8}{dotted},colspec={X[c]X[c]X[c]X[c]X[c]X[c]X[c]X[c]},row{1}={gray9,fg=black},colsep=0pt} 수학 & 물리학 & 화학 & 생명과학 & 지구과학 & 정보 & 공학 & 융합 \\ & & & & & √ & & \end{tblr}} & & & \\ \hline
		연구비 & 3,200,000원 & & & \\ \hline
        연구 기간 & 2025.00.00. \textasciitilde \ 2025.00.00. & & & \\ \hline
        연구 참여자 & 지도교사 & {\begin{tblr}{vline{2-4}={1-2}{dotted},hline{2}={1-4}{dotted},colspec={X[2.5,c]X[7,c]X[3.5,c]X[6,c]},cell{1-2}{1,3}={bg=gray9,fg=black},colsep=0pt} 성명 & & 담당교과 & \\ 연락처 & & e-mail & \end{tblr}} & & \\ \hline
        & 학생 & 성~~명 & 생년월일(6자리) & 성별 \\
		& & 홍길동 & 070101 & 남 \\
		& & 홍길동 & 070101 & 남 \\
		& & 홍길동 & 070101 & 남 \\
		& & 홍길동 & 070101 & 남 \\ \hline
		{\normalfont\addfontfeature{LetterSpace=-7.0}\fontsize{12}{12}\selectfont\\[-3ex] 관계 규정 등 제반사항 및 연구자 윤리를 준수하면서 본 사업을 성실히 수행하였습니다.\\[1.4em] 2025년 11월 10일 \\[1.4em] \raggedleft 지도교사 : \quad 정종광 \quad (인) \\ 참여학생 : \quad 홍길동 \quad (인) \quad 홍길동 \quad (인)\\ \quad 홍길동 \quad (인) \quad 홍길동 \quad (인)\\[1.5em] \centering{\fontsize{16}{16}\selectfont \bfseries한국과학창의재단이사장 귀하}} & & & & \\ \hline[1.5pt]
	\end{tblr}\medskip
	
	\newpage
\egroup
\bgroup

	\part{연구결과 요약문}
	\begin{tblr}{width=\textwidth,hlines,vlines,column{2,4}={font=\fontsize{11}{11}\selectfont},column{1,3}={font=\sffamily\bfseries\fontsize{11}{11}\selectfont},colspec={Q[gray9,c,wd=80pt]Q[l,wd=120pt]Q[gray9,c,wd=80pt]X[l]},row{1}={ht=30pt},row{2}={ht=30pt},row{3}={ht=110pt},row{4}={ht=220pt},row{5}={ht=120pt},row{6}={ht=100pt},rows={m},cell{2-6}{2}={c=3}{l}}
        {학교명} & & {과제 관리번호} & \\
		{과~제~명} & {국문 제목} & & \\
		{연구 필요성\\및 목적} & {연구의 필요성과 목적 작성} & & \\
		{연구내용\\및\\연구결과} & {연구내용과 결과 작성} & & \\
		{연구\\기대효과} & {향후 연구 계획 작성} & & \\
		{주제어/핵심어\\(5개 이내)} & {Minesweeper, Algorithm X, Exact Cover} & & \\
	\end{tblr}
    
	\newpage
\egroup
\bgroup
	
	\part{과제 수행 과정}
	\DefTblrTemplate{contfoot-text}{default}{}
	\DefTblrTemplate{conthead-text}{default}{}
	\DefTblrTemplate{caption}{default}{}
	\DefTblrTemplate{conthead}{default}{}
	\DefTblrTemplate{capcont}{default}{}
	\begin{longtblr}{width=\textwidth,hlines,vlines,column{2}={font=\fontsize{11}{11}\selectfont},column{1}={font=\sffamily\bfseries\fontsize{11}{11}\selectfont},hline{3-6}={2-3}{dotted},vline{3}={3-6}{dotted},colspec={Q[gray9,c,wd=60pt]Q[l,wd=50pt]X[l]},cell{1-2,7}{2}={c=2}{l},cell{2}{1}={r=5}{c},row{1}={ht=160pt},row{2}={ht=100pt},row{3-6}={ht=40pt},row{7}={ht=100pt},rows={m},hspan=minimal}
		{전문가\\피드백 활용\\과정} & {전문가 피드백 작성} & \\
		{문제해결\\과정} & {문제해결 과정 작성} & \\
		{문제해결\\과정} & 팀원1 & {팀원1 작성}\\
		{문제해결\\과정} & 팀원2 & {팀원2 작성}\\
		{문제해결\\과정} & 팀원3 & {팀원3 작성}\\
		{문제해결\\과정} & 팀원4 & {팀원4 작성}\\
		{과제 수행\\과정에서의\\학습} & {과제 수행 과정에서의 학습 작성} & \\
	\end{longtblr}
	\newpage
\egroup
%{\dotum\hdotum\fontsize{9}{9}\selectfont\color{bluefg}~【서식 4.2.5.】~\_보고서 본문}%
	\part{연구결과}
	\bigskip\bigskip
\bgroup
	\setmainhangulfont{HumanMyeongjo.ttf}[Path=./fonts/,AutoFakeBold=1.5]
	\setmainfont{TIMES}[Path=./fonts/,Extension=.ttf,BoldFont=*BD,ItalicFont=*I,BoldItalicFont=*BI]
	\begin{tblr}{width=\textwidth,colspec={X[c]},row{1}={font=\bfseries\fontsize{16}{16}\selectfont},row{4}={font=\fontsize{12}{12}\selectfont\bfseries},row{2,5}={font=\color{gray}},row{3,6}={font=\fontsize{9}{9}\selectfont\color{gray}}}
		{국문 제목} \\[.5ex]
		{연구자1·연구자2·연구자3·연구자4}\\[-1em]
		{과학영재학교 경기과학고등학교}\\[1ex]
		{영문 제목}\\[.5ex]
		{John Doe·John Doe·John Doe·John Doe}\\[-1em]
		Gyeonggi Science High School for the Gifted\\
	\end{tblr}
	\setmainhangulfont{HANBatang.ttf}[Path=./fonts/,BoldFont=*B,WordSpace={1,0.2,0.2}]
	\xetexkofontregime{latin}[puncts=prevfont,colons=prevfont,hyphens=latin,parens=hangul,cjksymbols=hoveangul]
\egroup
\begin{abstract}
	 국문 초록은 400자 내외로 작성한다.\\
	\emph{[주제어: 5개 이내 작성]}
\end{abstract}
\begin{paper}

\section{서론}
청소년 과학창의연구 학술지는 과학(예술)영재학교 및 과학고등학교의 재학생이면 누구나 투고할 수 있는 학술지이다.
\section{이론적 배경}
\section{연구 내용}
\section{실험 및 결과}
\section{결론}

\section*{■ 참고문헌}
\renewcommand{\refname}{}
\vspace*{-3em}
\begin{thebibliography}{9}
    \bibitem{np-complete}
    Kaye, R. Minesweeper is np-complete. \textit{Mathematical Intelligencer} \textbf{22}, 9–15 (2000).
    \bibitem{knuthX}
    Knuth, D. E. Dancing links. \textit{arXiv preprint cs/0011047} (2000).
\end{thebibliography}
\end{paper}
\bgroup
\xetexkofontregime{hangul}[puncts=hangul,colons=hangul,hyphens=hangul,parens=hangul,cjksymbols=hangul,alphs=hangul]
    %{\dotum\hdotum\fontsize{9}{9}\selectfont\color{bluefg}~【서식 4.2.6.】~\_보고서 본문}\medskip\\
    {~\fontsize{11}{11}\selectfont【별첨 1】\bigskip\medskip}
    \begin{center}
    \bfseries
    {\fontsize{16}{16}\selectfont 연구윤리 체크리스트}{\fontsize{12}{12}\selectfont (학생용)\smallskip}
    \end{center}
    \begin{tblr}{width=\textwidth,hlines,vlines,colspec={X[2,c,m]X[8,l,m]X[1,c,m]X[1,c,m]},columns={font=\bfseries\fontsize{11}{11}\selectfont},row{1}={ht=20pt,gray9,c,font=\sffamily\bfseries\fontsize{11}{11}\selectfont},column{1}={font=\sffamily\bfseries\fontsize{11}{11}\selectfont},cell{4}{1}={r=3}{c},cell{8}{1}={r=2}{c},stretch=1.1}
    구분 & 내용 & 예 & 아니요 \\
    저자 & 저자는 모둠 구성원이 함께 논의하여 결정하였고, 기여한 사람만을 정직하게 제시했다. & \rlap{$\checkmark$}$\square$ & $\square$ \\
    {자료조사\\/출처표기} & 보고서에 다른 사람의 자료를 인용한 부분과 출처를 명확하게 알 수 있도록 표기하였다. & \rlap{$\checkmark$}$\square$ & $\square$ \\
    {연구과정\\/결론} & 모든 데이터는 연구노트에 직접 기록하였으며, 실수를 포함하여 연구 과정에 일어난 모든 상황을 정직하게 기록하였다. & \rlap{$\checkmark$}$\square$ & $\square$ \\
    & 실험에서 얻지 않은 데이터는 결과에 포함하지 않았다. & \rlap{$\checkmark$}$\square$ & $\square$ \\
    & 실험에서 얻은 데이터를 임의로 고치거나 누락하지 않았다. & \rlap{$\checkmark$}$\square$ & $\square$ \\
    참고문헌 & 참고문헌 목록에는 본문에 인용한 글의 목록을 모두 제시하였다. & \rlap{$\checkmark$}$\square$ & $\square$ \\
    중복게재 & 같은 내용의 연구 결과를 다른 기관 또는 대회에 제출하지 않았다. & \rlap{$\checkmark$}$\square$ & $\square$ \\
    & 이전 연구결과와 동일 또는 실질적으로 유사한 연구내용 또는 저작물을 출처표시 없이 활용하지 않았다. & \rlap{$\checkmark$}$\square$ & $\square$ \\
    \end{tblr}
    \vskip8ex
    \begin{center}
    \bfseries
    {\fontsize{12}{12}\selectfont 제출한 보고서에 대해 연구윤리 준수여부를 성실히 이행하였음을 확인합니다.}\\[3em]
    \fontsize{13}{13}\selectfont 2025년 11월 10일\\[3em]
    \end{center}
    \begin{flushright}
    \bfseries
    {\fontsize{14}{14}\selectfont 연구참여자 {\color{brownfg} 홍길동}~~(인)~~{\color{brownfg} 홍길동}~~(인)~~{\color{brownfg} 홍길동}~~(인)~~{\color{brownfg} 홍길동}~~(인)}\\[1em]
    {\color{brownfg}\fontsize{11}{11}\selectfont (R\&E 참여학생 전원 서명)}\\[4em]
    \end{flushright}
    
    {\sffamily\addhangulfontfeature{InterHangul=-0.1em}\selectfont\color{brownfg} ※ (주의) 평가과정에서 연구윤리 위반 사항이 드러나는 경우 연구과제 수행 취소 및 연구비가 회수될 수 있음.}
\egroup

\newpage
\bgroup
\xetexkofontregime{hangul}[puncts=hangul,colons=hangul,hyphens=hangul,parens=hangul,cjksymbols=hangul,alphs=hangul]
    %{\dotum\hdotum\fontsize{9}{9}\selectfont\color{bluefg}~【서식 4.2.6.】~\_보고서 본문}\medskip\\
    {~\fontsize{11}{11}\selectfont【별첨 2】\bigskip\medskip}
    \begin{center}
    \bfseries
    {\fontsize{16}{16}\selectfont R\&E 지도 체크리스트}{\fontsize{12}{12}\selectfont (지도교사용)\smallskip}
    \end{center}
    \begin{tblr}{width=\textwidth,hlines,vlines,colspec={X[2.2,c,m]X[7.8,l,m]X[1,c,m]X[1,c,m]},columns={font=\bfseries\fontsize{11}{11}\selectfont},row{1}={ht=20pt,gray9,c,font=\sffamily\bfseries\fontsize{11}{11}\selectfont},row{2-4}={ht=60pt},row{5}={ht=180pt},column{1}={font=\sffamily\bfseries\fontsize{11}{11}\selectfont},cell{5}{1}={c=4}{l},stretch=1.1}
    구분 & 내용 & 예 & 아니요 \\
    연구보고서 & 제목, 초록, 연구 필요성 및 목적, 연구문제 및 연구방법, 연구결과 및 해석, 참고문헌이 적절하게 작성되었는지 확인하였는가? & \rlap{$\checkmark$}$\square$ & $\square$ \\
    연구윤리 & 표절이나 불필요한 저자 포함이나 실험 데이터 조작 등 윤리적 문제에 저촉되지 않게 지도하였는가? & \rlap{$\checkmark$}$\square$ & $\square$ \\
    실험안전 & 모든 연구과정에서 안전을 확보하고 안전교육을 진행하였는가? & \rlap{$\checkmark$}$\square$ & $\square$ \\
    {지도교사 의견\\[1.5em]
     \color{brownfg} 지도교사 의견 작성}\\
    \end{tblr}
    \vskip8ex
    \begin{center}
    \bfseries
    {\fontsize{12}{12}\selectfont 위 체크리스트를 사실에 기반하여 작성하였음을 확인합니다.
}\\[3em]
    \fontsize{13}{13}\selectfont 2025년 11월 10일\\[3em]
    \end{center}
    \begin{flushright}
    \bfseries
    {\fontsize{14}{14}\selectfont 지도교사 {\color{brownfg} 정종광}~~(인)}\\[1em]
    \end{flushright}
\egroup
\end{document}